\documentclass{note}

\title{Physics Notes}
\author{LI, Yu Hong Harry}

\begin{document}
\maketitle

\chapter{Work and Kinetic Energy}

The concepts of \textit{Work and Energy} is introduced because of the problem of varying force. Furthermore, the reason that work and energy are chosen is that they obey \textit{the principle of energy conservation}.

\chapter{Wave}

A \textit{wave} is an \textit{oscillation} that transfers energy to the direction of wave propagation. A \textit{mechanical wave} only transfers energy but not mass.

\paragraph{Transverse Wave} is a wave whose local mass motion is perpendicular to the direction of wave propagation.

\paragraph{Longitudinal Wave} is a wave whose local mass motion is parallel to the direction of wave propagation.

\section{Wave Function}

Wave function \(y = f(x, t)\) describes the displacement \(y\) from the equilibrium position at a certain position \(x\) and a certain time \(t\).

The wave function of an arbitrary sinusoidal wave is
\[
  y = A \cos(k x - \omega t)
\]
where \(k\) is the \textit{wave number}, defined by
\[
  k = \frac{2\pi}{\lambda} \quad \mathbf{(rad/m)}
\]
and \(\omega\) is the \textit{angular frequency}, defined by
\[
  \omega = 2\pi f = \frac{2\pi}{T} \quad \mathbf{(rad/s)}
\]
Combining the wave number and the angular frequency,
\[
  v = f \lambda = \frac{\omega}{k}
\]

\section{Wave Equation}

In a transverse wave, consider a tiny segment \(\Delta x\) of the string. Since the tiny segment has neither horizontal displacement, velocity nor acceleration, the net horizontal force \(F_x = 0\). Since the tiny segment may have vertical displacement, velocity or acceleration, the net vertical force \(F_y\) does not always equal to \(0\).

Let \(x\) and \(x + \Delta x\) be the two endpoints of the segment. By the decomposition of force, consider the vertical forces \(F_{y}(x)\) and \(F_{y}(x+\Delta x)\),
\begin{align*}
  F_y(x)            & = \pm F_x \eval{\dv{y}{x}}_x              \\
  F_y(x + \Delta x) & = \pm F_x \eval{\dv{y}{x}}_{x + \Delta x}
\end{align*}
Since \(F_y(x)\) and \(F_y(x + \Delta x)\) are always in different direction, their symbols are always opposite. Let \(F_y(x + \Delta x)\) be positive and \(F_y(x)\) be negative, so that the following steps will be simpler.

The net force in y-direction
\begin{align*}
  F_y
    & = F_y(x) + F_y(x + \Delta x)                                  \\
    & = F_x \left(\dv{y}{x}\/(x + \Delta x) - \dv{y}{x}\/(x)\right)
\end{align*}
and since
\begin{align*}
  F_y
    & = ma_y                      \\
    & = \Delta x \mu \dv[2]{y}{t}
\end{align*}
therefore
\begin{align*}
  F_x \left(\dv{y}{x}\/(x + \Delta x) - \dv{y}{x}\/(x)\right)                & = \Delta x \mu \dv[2]{y}{t}    \\
  \frac{1}{\Delta x} \left(\dv{y}{x}\/(x + \Delta x) - \dv{y}{x}\/(x)\right) & = \frac{\mu}{F_x} \dv[2]{y}{t} \\
  \dv[2]{y}{x}                                                               & = \frac{\mu}{F_x} \dv[2]{y}{t}
\end{align*}

By substituting the wave function \(y = A \cos(k x - \omega t)\), the (partial, since wave function is a multi-variable function) derivatives become
\begin{align*}
  \pdv[2]{y}{x} & = -k^2 A \cos(k x - \omega t) \\
  \pdv[2]{y}{t} & = -\omega^2 A \cos(k x - \omega t)
\end{align*}
and the wave equation become
\begin{align*}
      -k^2 A \cos(k x - \omega t) & = \frac{\mu}{F_x} (-\omega^2 A \cos(k x - \omega t)) \\
      k^2 & = \frac{\mu}{F_x} \omega^2 \\
      \frac{1}{v^2} & = \frac{\mu}{F_x} \\
      v & = \sqrt{\frac{F_x}{\mu}} \\
\end{align*}

By applying back the second last equation, the wave equation can also be rewritten as
\[
\dv[2]{y}{x} = \frac{1}{v^2} \dv[2]{y}{t}
\]

\section{Power Propagation}

The intensity of power is
\[
  I = \frac{P}{4 \pi r^2}
\]

\section{Reflection}

The incident wave
\[
  y(x, y) = A\cos(kx-\omega t)
\]

Open boundary condition
\[
  y(x, y) = +A\cos(kx+\omega t)
\]

The fixed boundary condition
\[
  y(x, y) = -A\cos(kx+\omega t)
\]

\section{Standing Wave}

The standing wave is a superposition of 2 identical waves traveling in opposite directions.
Reflection is a way to generate a wave traveling in the opposite direction.
\begin{align*}
  y(x, t)
    & = A\cos(kx-\omega t) + A\cos(kx+\omega t) \\
    & = 2A\cos(kx)cos(\omega t)
\end{align*}
or
\begin{align*}
  y(x, t)
    & = A\cos(kx-\omega t) - A\cos(kx+\omega t) \\
    & = 2A\sin(kx)sin(\omega t)
\end{align*}
Therefore, there is also no wave propagation and power propagation because \(kx-\omega t\) doesn't appear!

\subsection{Node \& Antinode}

A \textit{node} (N) is where the wave has zero amplitude due to
\[
  \cos(kx) = 0
\]
This is called destructive interference.

An \textit{antinode} (A) is where the wave achieves maximum amplitude due to
\[
  \cos(kx) = 1
\]
This is called constructive interference.

The allowed normal mode wavelength can be
\[
  L = n\frac{\lambda}{2}
\]
or
\[
  \lambda = \frac{2L}{n}
\]
and the normal mode frequencies are
\[
  f_n = n\frac{v}{2L} = nf_1
\]


\section{Beats}

Beats are interference of waves traveling in the same direction with different frequencies.

\[
  y_a + y_b = 2A\cos(-2\pi \frac{f_a + f_b}{2}t) \cos(-2\pi \frac{f_a - f_b}{2}t)
\]

There are two beats per period, since there are two frequencies
\begin{align*}
  f  & = \frac{f_a + f_b}{2}   \\
  f' & = \frac{|f_a - f_b|}{2}
\end{align*}

\section{Doppler Effect}

The Doppler effect causes the frequency of a traveling wave to change when the source and / or observer are moving.
\begin{align*}
  f_L
    & = \frac{v + v_L}{\lambda} \\
    & = (1 + \frac{v_L}{v})f_s
\end{align*}

Considering there is a relative velocity \(v_s\)
\begin{align*}
  \lambda = \frac{v \pm v_s}{f_s}
\end{align*}

\chapter{Thermodynamics}

\textit{Thermodynamics} is the branch of physics concerned with \textit{heat} and \textit{temperature} and their relation to \textit{energy} and \textit{work}.

The behaviour of these quantities is governed by the \textit{Four Laws of Thermodynamics}. The laws are explained in terms of microscopic constituents by \textit{Statistical Mechanics}.

\section{Temperature \& Heat}

\paragraph{Temperature} is briefly a measure of hotness or coldness. Moreover, it is a measure of tendency for an object to spontaneously give up its internal energy to its surroundings. An object with high temperature tend to give more energy to its surroundings; an object with lower temperature tend to give less energy to its surroundings.

\begin{tip}
    A definition of \textit{temperature} will be given in later sections.
\end{tip}

\paragraph{Thermal Equilibrium} is the state if the interaction between the objects in system causes no further change in the system. Systems are in thermal equilibrium if and only if they have the \textbf{same temperature}.

\paragraph{The 0-th Law of Thermodynamics} states that, if $C$ is initially in thermal equilibrium with both $A$ and $B$, then $A$ and $B$ are also in thermal equilibrium with each other.

That is, $\qty{0}{K} = \qty{273.15}{\degreeCelsius}$

\[
  \frac{1}{T} = \dv{S}{U}
\]

\paragraph{Celsius}
The freezing and boiling points of water at \(1 \;\text{atm}\) are \(0 \;\celsius\)

\paragraph{Kelvin}

\paragraph{Constant Volume gas Thermometer}

Recall the ideal gas law
\[
  pV = nRT
\]
\[
  p \propto T
\]

\[
  T = T_\text{triple}\frac{p}{p_\text{triple}}
\]

\section{The 0-th Law of Thermodynamics}


\section{Thermal Expansion}

\[
  \frac{\Delta L}{L} = \alpha \Delta T
\]
That is, the fractional change in the length \(\frac{\Delta L}{L}\) is proportional to the temperature change \(\Delta T\).

\[
  \frac{\Delta V}{V} = \beta \Delta T
\]

\begin{align*}
  V = L^3
  \dd{V} = 3L^2 \dd{L}
  \beta = \frac{1}{\Delta T}\dv{}{}
\end{align*}

\section{Heat}

\textit{Heat} is energy transferred from on body to another body due to temperature difference. How do we quantify heat?

\paragraph{Heat Capacity} is the relation ship between the head and the temperature change, without changing the form.

\paragraph{Latent Heat} is the amount of heat needed to convert a unit mass of substance from one phase to another. This change of phase occurs without temperature change.

\section{Thermal Properties}

\paragraph{State Variables} is the state of matter in terms of macroscopic quantities, such as pressure \(p\), volume \(V\), quantity \(n\) or \(N\), temperature \(T\).

\subsection{Ideal Gas}

Experiments have shown that the ideal gases with the same quantity $\frac{pV}{n}$ have the same temperature $T$. Therefore, the simplest choice will be
\begin{equation*}
    \frac{pV}{n} \propto T
\end{equation*}

By introduce a constant $R$, the equation becomes
\begin{equation*}
    pV = nRT
\end{equation*}

\paragraph{Equation of State}

\paragraph{State Function}

\section{Heat Capacity of an Ideal Gas}

\chapter{The 1-th Law of Thermodynamics}

\paragraph{Thermodynamic Process} changes the state of a thermodynamic system, defined by state variables, from the initial state to the final state along a sequence of intermediate states called the path.

The change of a state function depends and only depends on the initial and final values of the state variables. If and only if a quantity depends on the path, it's a state function.

\paragraph{Thermodynamic System} is one that may exchange energy with the environment, in form of work done by the system and/or heat added to the system. The sign conventions are:
\begin{itemize}
  \item \(Q > 0\) if heat added to the system.
  \item \(Q < 0\) if heat subtracted from the system.
  \item \(W > 0\)
  \item \(W < 0\)
\end{itemize}

The first law of thermodynamics is a statement about the conservation of energy.
\begin{align*}
  Q      & = \Delta U + W    \\
  \dd{Q} & = \dd{U} + \dd{W}
\end{align*}

\section{Work Done by Ideal Gas: Path Dependence}

\[
  W = \int_{V}^{V'} p\dd{V}
\]

If the pressure is constant (isobaric)
\[
  W = \int_{V}^{V'} p\dd{V} = p(V'-V)C
\]

A constant temperature (isothermal)
\begin{align*}
  W & = \int_{V}^{V'} p\dd{V} = \int_{V}^{V'} \frac{nRT}{V}\dd{V} \\
    & = nRT\ln\frac{V'}{V}                                        \\
    & = nRT\ln\frac{p'}{p}
\end{align*}

Internal Energy: State Function


\section{Heat Added to a Gas: Path Dependence}

\chapter{First Law of Thermodynamics II}

\paragraph{Isochoric} means that through this process, the volume change $\Delta V$ is zero.

\paragraph{Isobaric} means that the pressure change $\Delta p$ is zero.

\paragraph{} means that the heat change is zero.

\section{Adiabatic Process} 
An adiabatic process is one in which there is no workflow, $Q = 0$. This may happen if: 
\begin{itemize}
    \item The system is thermally isolated so that heat cannot either enter or leave.
    \item The process takes place quickly compared to the rate of heat flow, so that the process is finished before heat has time enough to enter or leave the system.
\end{itemize}

\paragraph{Adiabatic Compression}
\(Q = 0\) and \(W = p\Delta V < 0\), \(\Delta U = Q - W > 0\)

\paragraph{Ad8iabatic Expansion}
\(Q = 0\) and \(W = p\Delta V > 0\), \(\Delta U = Q - W < 0\)

\chapter{Second Law of Thermodynamics}

\section{Reversible \& Irreversible Process}

Reversible process is a thermodynamic process that the direction can be reversed by making infinitesimal change in condition. Throughout the process, the system is infinitesimally close to equilibrium.

\section{Carnot Cycle}

\section{Absolute Temperature}


\end{document}